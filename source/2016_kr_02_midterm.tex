\element{2016_kr_02_midterm}{ % в фигурных скобках название группы вопросов
%  %\AMCnoCompleteMulti
\begin{questionmult}{1} % тип вопроса (questionmult — множественный выбор) и в фигурных — номер вопроса
В регрессии с константой и тремя объясняющими переменными
сумма квадратов остатков равна $162$,
а число наблюдений равно $31$.
Точечная оценка дисперсии случайной составляющей равна
\begin{multicols}{3} % располагаем ответы в 3 колонки
\begin{choices} % опция [o] не рандомизирует порядок ответов
       \correctchoice{$6$}
       \wrongchoice{$7$}
       \wrongchoice{$8$}
       \wrongchoice{$2.45$}
       \wrongchoice{$2.65$}
       \wrongchoice{$2.83$}
    \end{choices}
   \end{multicols}
\end{questionmult}
}


\element{2016_kr_02_midterm}{ % в фигурных скобках название группы вопросов
%  %\AMCnoCompleteMulti
\begin{questionmult}{2} % тип вопроса (questionmult — множественный выбор) и в фигурных — номер вопроса
Если для регрессора используется преобразование Бокса-Кокса с параметром $\theta=-1$,
а для зависимой переменной — с параметром $\lambda = 1$,
то регрессионное уравнение представимо в виде
\begin{multicols}{3} % располагаем ответы в 3 колонки
\begin{choices} % опция [o] не рандомизирует порядок ответов
       \correctchoice{$Y_i = \beta_1 + \beta_2 \frac{1}{X_i} + u_i$}
       \wrongchoice{$\ln Y_i = \beta_1 + \beta_2 \ln X_i + u_i$}
       \wrongchoice{$\ln Y_i = \beta_1 + \beta_2 X_i + u_i$}
       \wrongchoice{$Y_i = \beta_1 + \beta_2 \ln X_i + u_i$}
       \wrongchoice{$Y_i = \beta_1 + \beta_2 X_i + u_i$}
       \wrongchoice{$\ln Y_i = \beta_1 - \beta_2 \ln X_i + u_i$}
    \end{choices}
   \end{multicols}
\end{questionmult}
}

\element{2016_kr_02_midterm}{ % в фигурных скобках название группы вопросов
%  %\AMCnoCompleteMulti
\begin{questionmult}{3} % тип вопроса (questionmult — множественный выбор) и в фигурных — номер вопроса
Известно, что регрессоры $X$ и $Z$ ортогональны,
а истинная зависимость описывается уравнением
$Y_i = \alpha_1 + \alpha_2 X_i + \alpha_3 Z_i + u_i$.
Исследователь оценивает с помощью МНК две регрессии:
$\hat Y_i = \hb_1 + \hb_2 X_i$ и $\hat Y_i = \hat \gamma_1 + \hat \gamma_2 Z_i$.
При этом
%\begin{multicols}{3} % располагаем ответы в 3 колонки
\begin{choices} % опция [o] не рандомизирует порядок ответов
       \correctchoice{$\hb_2$ — несмещённая оценка для $\alpha_2$; $\hat \gamma_2$ — несмещённая оценка для $\alpha_3$}
       \wrongchoice{$\hb_2$ — смещённая оценка для $\alpha_2$; $\hat \gamma_2$ — смещённая оценка для $\alpha_3$}
       \wrongchoice{$\hb_2$ — несмещённая оценка для $\alpha_2$; $\hat \gamma_2$ — смещённая оценка для $\alpha_3$}
       \wrongchoice{$\hb_2$ — смещённая оценка для $\alpha_2$; $\hat \gamma_2$ — несмещённая оценка для $\alpha_3$}
       \wrongchoice{$\hb_2$ — эффективная оценка для $\alpha_2$; $\hat \gamma_2$ — эффективная оценка для $\alpha_3$}
    \end{choices}
   %\end{multicols}
\end{questionmult}
}


\element{2016_kr_02_midterm}{ % в фигурных скобках название группы вопросов
%  %\AMCnoCompleteMulti
\begin{questionmult}{4} % тип вопроса (questionmult — множественный выбор) и в фигурных — номер вопроса
Гипотеза о том, что одновременно $\beta_1 + \beta_2 =1$  и $\beta_3=0$
в множественной линейной регрессии построенной по $n$ наблюдениям
проверяется с помощью  статистики, имеющей распределение
\begin{multicols}{3} % располагаем ответы в 3 колонки
\begin{choices} % опция [o] не рандомизирует порядок ответов
       \correctchoice{$F$}
       \wrongchoice{$t_n$}
       \wrongchoice{$t_{n-2}$}
       \wrongchoice{$t_{n-k}$}
       \wrongchoice{$N(0;1)$}
       \wrongchoice{Демешева-Мамонтова}
    \end{choices}
   \end{multicols}
\end{questionmult}
}

\element{2016_kr_02_midterm}{ % в фигурных скобках название группы вопросов
%  %\AMCnoCompleteMulti
\begin{questionmult}{5} % тип вопроса (questionmult — множественный выбор) и в фигурных — номер вопроса
Элеонора исследует зависимость цены номера в отеле от звёздности отеля, $star$,
(от 1 до 3 звёзд) и расстояния до моря, $dist$.
Элеонора хочет оценить модель вида
$price_i = \beta_1 + \beta_2 star_i + \beta_3 dist_i + u_i$.
Чтобы считаться богиней эконометрики Элеоноре стоит
\begin{multicols}{1} % располагаем ответы в 3 колонки
\begin{choices} % опция [o] не рандомизирует порядок ответов
        \correctchoice{заменить переменную $star_i$ на дамми-переменные $one_i$ и $two_i$, равные 1 для отелей с одной и двумя звёздами соответственно}
       \wrongchoice{использовать МНК для оценки данной модели}
       \wrongchoice{добавить в модель переменную $z_i = star_i \cdot dist_i$}
       \wrongchoice{добавить в модель переменную $z_i = star^2_i$, так как эффект звёздности наверняка нелинейный}
       \wrongchoice{заменить переменную $star_i$ на дамми-переменные $one_i$, $two_i$ и $three_i$, равные 1 для отелей с одной, двумя и тремя звёздами соответственно}
        \wrongchoice{добавить дамми-переменные $one_i$, $two_i$ и $three_i$, равные 1 для отелей с одной, двумя и тремя звёздами соответственно}
    \end{choices}
   \end{multicols}
\end{questionmult}
}


\element{2016_kr_02_midterm}{ % в фигурных скобках название группы вопросов
%  %\AMCnoCompleteMulti
\begin{questionmult}{6} % тип вопроса (questionmult — множественный выбор) и в фигурных — номер вопроса
Показатель $R^2_{adj}$ можно вычислить по формуле
\begin{multicols}{3} % располагаем ответы в 3 колонки
\begin{choices} % опция [o] не рандомизирует порядок ответов
       \correctchoice{$R^2_{adj} = (-1)\cdot \frac{k-1}{n-k} + R^2 \cdot \frac{n-1}{n-k}$}
       \wrongchoice{$R^2_{adj} = \frac{k-1}{n-k} + R^2 \cdot \frac{n-1}{n-k}$}
       \wrongchoice{$R^2_{adj} = \frac{k-1}{n-k} - R^2 \cdot \frac{n-1}{n-k}$}
       \wrongchoice{$R^2_{adj} = \frac{k-1}{n-k} + R^2 \cdot \frac{n-k}{n-1}$}
       \wrongchoice{$R^2_{adj} = \frac{k}{n-k} + R^2 \cdot \frac{n-1}{n-k}$}
       \wrongchoice{$R^2_{adj} = \frac{n-k}{k-1} + R^2 \cdot \frac{n-1}{n-k}$}
    \end{choices}
   \end{multicols}
\end{questionmult}
}

\element{2016_kr_02_midterm}{ % в фигурных скобках название группы вопросов
%  %\AMCnoCompleteMulti
\begin{questionmult}{7} % тип вопроса (questionmult — множественный выбор) и в фигурных — номер вопроса
Если гипотеза $\beta_2 + \beta_3 = 1$ верна, то модель $\ln Y_i = \beta_1 + \beta_2 \ln X_i + \beta_3 \ln Z_i + u_i$ совпадает с моделью
\begin{multicols}{2} % располагаем ответы в 3 колонки
\begin{choices} % опция [o] не рандомизирует порядок ответов
       \correctchoice{$\ln (Y_i/Z_i) = \beta_1 + \beta_2 \ln (X_i/Z_i) + u_i $}
       \wrongchoice{$\ln Y_i = \beta_1 + \beta_2 \ln (X_i/Z_i) + u_i $}
       \wrongchoice{$\ln Y_i = \beta_1 + \beta_2 \ln (Z_i/Y_i) + u_i $}
       \wrongchoice{$\ln (Y_i/Z_i) = \beta_1 + \beta_2 \ln (Y_i/Z_i) + u_i $}
       \wrongchoice{$\ln (Y_i/Z_i) = \beta_1 + \beta_2 \ln (Y_i/X_i) + u_i $}
    \end{choices}
   \end{multicols}
\end{questionmult}
}


\element{2016_kr_02_midterm}{ % в фигурных скобках название группы вопросов
%  %\AMCnoCompleteMulti
\begin{questionmult}{8} % тип вопроса (questionmult — множественный выбор) и в фигурных — номер вопроса
Гипотеза о неадекватности множественной регрессии проверяется с помощью статистики равной
\begin{multicols}{3} % располагаем ответы в 3 колонки
\begin{choices} % опция [o] не рандомизирует порядок ответов
       \correctchoice{$\frac{ESS/(k-1)}{RSS/(n-k)}$}
       \wrongchoice{$\frac{\hb - \beta}{se(\hb)}$}
       \wrongchoice{$\frac{TSS/(n-1)}{RSS/(n-k)}$}
       \wrongchoice{$\frac{TSS/(n-1)}{ESS/(k-1)}$}
       \wrongchoice{$\frac{ESS}{TSS}$}
       \wrongchoice{$\frac{RSS}{TSS}$}
    \end{choices}
   \end{multicols}
\end{questionmult}
}


\element{2016_kr_02_midterm}{ % в фигурных скобках название группы вопросов
%  %\AMCnoCompleteMulti
\begin{questionmult}{9} % тип вопроса (questionmult — множественный выбор) и в фигурных — номер вопроса
Исследователь выполнил второй шаг в PE-тесте МакКиннона. В регрессии $\ln Y_i$ на исходные регрессоры и $Z_i = \hat Y_i - \exp(\widehat{\ln Y_i})$ коэффициент при $Z_i$ оказался значимым. А в регрессии $Y_i$ на исходные регрессоры и $W_i = \ln \hat Y_i - \widehat{\ln Y_i}$ коэффициент при $W_i$ оказался незначимым. Из результатов следует сделать вывод, что
\begin{multicols}{3} % располагаем ответы в 3 колонки
\begin{choices} % опция [o] не рандомизирует порядок ответов
       \correctchoice{следует предпочесть линейную модель}
       \wrongchoice{следует предпочесть полулогарифмеческую модель}
       \wrongchoice{следует предпочесть логарифмическую модель}
       \wrongchoice{тесты противоречат друг другу, ни одна из моделей не предпочитается}
       \wrongchoice{в исходной модели пропущен регрессор $Z_i$}
       \wrongchoice{в исходной модели пропущен регрессор $W_i$}
    \end{choices}
   \end{multicols}
\end{questionmult}
}


\element{2016_kr_02_midterm}{ % в фигурных скобках название группы вопросов
%  %\AMCnoCompleteMulti
\begin{questionmult}{10} % тип вопроса (questionmult — множественный выбор) и в фигурных — номер вопроса
Истинной является модель $Y_i = \beta_1 + \beta_2 X_i + u_i$. Глафира оценивает две регрессии: $\hat Y_i = \hb_1 + \hb_2X_i$ и  $\hat Y_i = \hat \gamma_1 + \hat \gamma_2 X_i + \hat \gamma_3 Z_i$ с помощью МНК. Для коэффициента $\beta_2$
\begin{multicols}{2} % располагаем ответы в 3 колонки
\begin{choices} % опция [o] не рандомизирует порядок ответов
       \correctchoice{оценки $\hb_2$ и $\hat\gamma_2$ являются несмещёнными}
       \wrongchoice{оценки $\hb_2$ и $\hat\gamma_2$ являются эффективными}
       \wrongchoice{оценка $\hb_2$ является несмещённой, а оценка  $\hat\gamma_2$ — смещённой}
       \wrongchoice{оценки $\hb_2$ и $\hat\gamma_2$ являются неэффективными}
       \wrongchoice{оценка $\hb_2$ является смещённой, а оценка  $\hat\gamma_2$ — несмещённой}
    \end{choices}
   \end{multicols}
\end{questionmult}
}
