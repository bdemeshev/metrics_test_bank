\usepackage{fontspec}
\usepackage{polyglossia}
\usepackage{csquotes}

\setmainlanguage{russian}
\setotherlanguages{english}

% download "Linux Libertine" OTF-fonts:
% http://www.linuxlibertine.org/index.php?id=91&L=1
\setmainfont{Linux Libertine O} % or Helvetica, Arial, Cambria
% why do we need \newfontfamily:
% http://tex.stackexchange.com/questions/91507/
\newfontfamily{\cyrillicfonttt}{Linux Libertine O}
\newfontfamily{\cyrillicfont}{Linux Libertine O}
\newfontfamily{\cyrillicfontsf}{Linux Libertine O}

\usepackage{etoolbox} % to use ifdef, must be after babel


\usepackage[paper=a4paper, top=13.5mm, bottom=13.5mm, left=16.5mm, right=13.5mm, includefoot]{geometry}

\usepackage{etex} % расширение классического tex
% в частности позволяет подгружать гораздо больше пакетов, чем мы и займёмся далее

\usepackage{floatrow} % сильно вниз если сдвинуть - ругается!


\usepackage{makeidx} % для создания предметных указателей
\usepackage{verbatim} % для многострочных комментариев
%\usepackage[pdftex]{graphicx} % для вставки графики
% omit pdftex option if not using pdflatex


%\usepackage{dsfont} % шрифт для единички с двойной палочкой (для индикатора события)
\usepackage{bbm} % шрифт - двойные буквы


\usepackage[usenames, dvipsnames, svgnames, table, rgb]{xcolor}

\usepackage{colortbl}


% пакет для тестов:
\usepackage[box, % запрет на перенос вопросов
nopage, % убираем колонтитулы страницы
insidebox, % ставим буквы в квадратики
separateanswersheet, % добавляем бланк ответов
nowatermark, % отсутствие надписи "Черновик"
indivanswers,  % показываем верные ответы
%answers,
lang=RU, % локализация слов "нет верных ответов", "вопрос" и тд
completemulti % добавлять "нет правильного ответа" во всех вопросах множественного выбора
]{automultiplechoice}


\usepackage{multicol}
\usepackage{multirow} % Слияние строк в таблице


\usepackage[colorlinks, hyperindex, unicode, breaklinks]{hyperref} % гиперссылки в pdf

\usepackage{dcolumn}

\usepackage{amssymb}
\usepackage{amsmath}
\usepackage{amsthm}
\usepackage{epsfig}
\usepackage{bm}
\usepackage{color}

\usepackage{todonotes} % для вставки в документ заметок о том, что осталось сделать
% \todo{Здесь надо коэффициенты исправить}
% \missingfigure{Здесь будет картина Последний день Помпеи}
% команда \listoftodos — печатает все поставленные \todo'шки


\usepackage{textcomp}  % Чтобы в формулах можно было русские буквы писать через \text{}

%\usepackage{embedfile} % Чтобы код LaTeXа включился как приложение в PDF-файл

\usepackage{subfigure} % для создания нескольких рисунков внутри одного

\usepackage{tikz, pgfplots} % язык для рисования графики из latex'a
\usetikzlibrary{trees} % прибамбас в нем для рисовки деревьев
\usetikzlibrary{arrows} % прибамбас в нем для рисовки стрелочек подлиннее
\usepackage{tikz-qtree} % прибамбас в нем для рисовки деревьев




\usepackage{enumitem}


%\embedfile[desc={Исходный LaTeX файл}]{\jobname.tex} % Включение кода в выходной файл
%\embedfile[desc={Стилевой файл}]{title_bor_utf8.tex}



% вместо горизонтальной делаем косую черточку в нестрогих неравенствах
\renewcommand{\le}{\leqslant}
\renewcommand{\ge}{\geqslant}
\renewcommand{\leq}{\leqslant}
\renewcommand{\geq}{\geqslant}

% делаем короче интервал в списках
\setlength{\itemsep}{0pt}
\setlength{\parskip}{0pt}
\setlength{\parsep}{0pt}

% свешиваем пунктуацию (т.е. знаки пунктуации могут вылезать за правую границу текста, при этом текст выглядит ровнее)
\usepackage{microtype}

% более красивые таблицы
\usepackage{booktabs}
% заповеди из докупентации:
% 1. Не используйте вертикальные линни
% 2. Не используйте двойные линии
% 3. Единицы измерения - в шапку таблицы
% 4. Не сокращайте .1 вместо 0.1
% 5. Повторяющееся значение повторяйте, а не говорите "то же"


% DEFS
\def \mbf{\mathbf}
\def \msf{\mathsf}
\def \mbb{\mathbb}
\def \tbf{\textbf}
\def \tsf{\textsf}
\def \ttt{\texttt}
\def \tbb{\textbb}

\def \wh{\widehat}
\def \wt{\widetilde}
\def \ni{\noindent}
\def \ol{\overline}
\def \cd{\cdot}
\def \bl{\bigl}
\def \br{\bigr}
\def \Bl{\Bigl}
\def \Br{\Bigr}
\def \fr{\frac}
\def \bs{\backslash}
\def \lims{\limits}
\def \arg{{\operatorname{arg}}}
\def \dist{{\operatorname{dist}}}
\def \VC{{\operatorname{VCdim}}}
\def \card{{\operatorname{card}}}
\def \sgn{{\operatorname{sign}\,}}
\def \sign{{\operatorname{sign}\,}}
\def \xfs{(x_1,\ldots,x_{n-1})}
\def \Tr{{\operatorname{\mbf{Tr}}}}
\DeclareMathOperator*{\argmin}{arg\,min}
\DeclareMathOperator*{\argmax}{arg\,max}
\DeclareMathOperator*{\amn}{arg\,min}
\DeclareMathOperator*{\amx}{arg\,max}
\def \cov{{\operatorname{Cov}}}
\DeclareMathOperator{\Var}{Var}
\DeclareMathOperator{\Cov}{Cov}
\DeclareMathOperator{\Corr}{Corr}
\DeclareMathOperator{\grad}{grad}
\DeclareMathOperator{\trace}{trace}
\DeclareMathOperator{\rank}{rank}

\def \xfs{(x_1,\ldots,x_{n-1})}
\def \ti{\tilde}
\def \wti{\widetilde}


\def \mL{\mathcal{L}}
\def \mW{\mathcal{W}}
\def \mH{\mathcal{H}}
\def \mC{\mathcal{C}}
\def \mE{\mathcal{E}}
\def \mN{\mathcal{N}}
\def \mA{\mathcal{A}}
\def \mB{\mathcal{B}}
\def \mU{\mathcal{U}}
\def \mV{\mathcal{V}}
\def \mF{\mathcal{F}}

\def \R{\mbb R}
\def \N{\mbb N}
\def \Z{\mbb Z}
\def \P{\mbb{P}}
%\def \p{\mbb{P}}
\def \E{\mbb{E}}
\def \D{\msf{D}}
\def \I{\mbf{I}}

\def \a{\alpha}
\def \b{\beta}
\def \t{\tau}
\def \dt{\delta}
\def \e{\varepsilon}
\def \ga{\gamma}
\def \kp{\varkappa}
\def \la{\lambda}
\def \sg{\sigma}
\def \sgm{\sigma}
\def \tt{\theta}
\def \ve{\varepsilon}
\def \Dt{\Delta}
\def \La{\Lambda}
\def \Sgm{\Sigma}
\def \Sg{\Sigma}
\def \Tt{\Theta}
\def \Om{\Omega}
\def \om{\omega}


\def \ni{\noindent}
\def \lq{\glqq}
\def \rq{\grqq}
\def \lbr{\linebreak}
\def \vsi{\vspace{0.1cm}}
\def \vsii{\vspace{0.2cm}}
\def \vsiii{\vspace{0.3cm}}
\def \vsiv{\vspace{0.4cm}}
\def \vsv{\vspace{0.5cm}}
\def \vsvi{\vspace{0.6cm}}
\def \vsvii{\vspace{0.7cm}}
\def \vsviii{\vspace{0.8cm}}
\def \vsix{\vspace{0.9cm}}
\def \VSI{\vspace{1cm}}
\def \VSII{\vspace{2cm}}
\def \VSIII{\vspace{3cm}}



\newcommand{\dx}[1]{\,\mathrm{d}#1} % для интеграла: маленький отступ и прямая d
\newcommand{\ind}[1]{\mathbbm{1}_{\{#1\}}} % Индикатор события
%\renewcommand{\to}{\rightarrow}
\newcommand{\eqdef}{\mathrel{\stackrel{\rm def}=}}
\newcommand{\iid}{\mathrel{\stackrel{\rm i.\,i.\,d.}\sim}}
\newcommand{\const}{\mathrm{const}}

%на всякий случай пока есть
%теоремы без нумерации и имени
\newtheorem*{theorem}{Теорема}

%"Определения","Замечания"
%и "Гипотезы" не нумеруются
\newtheorem*{definition}{Определение}
%\newtheorem*{rem}{Замечание}
%\newtheorem*{conj}{Гипотеза}

%"Теоремы" и "Леммы" нумеруются
%по главам и согласованно м/у собой
%\newtheorem{theorem}{Теорема}
%\newtheorem{lemma}[theorem]{Лемма}

% Утверждения нумеруются по главам
% независимо от Лемм и Теорем
%\newtheorem{prop}{Утверждение}
%\newtheorem{cor}{Следствие}
