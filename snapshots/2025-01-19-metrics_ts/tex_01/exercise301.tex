
\begin{question}
Гюльчатай оценила одну и ту же модель зависимости зарплаты от опыта работы по трём разным группам наблюдений.

Для 1000 сельских жителей:
\[
wage_i = \beta_1 + \beta_2 exp_i + u_i, RSS_a = 150.
\]
Для 2004 городских жителей:
\[
wage_i = \gamma_1 + \gamma_2 exp_i + u_i, RSS_b = 200.
\]
Для всех жителей сразу:
\[
wage_i = \delta_1 + \delta_2 exp_i + u_i, RSS_c = 400.
\]

Найдите значение \(F\)-статистики теста Чоу для проверки гипотезы об одинаковой зависимости для всех жителей.

Ответ укажите с точностью до двух знаков после десятичной точки.
\end{question}

\begin{solution}
\[
F = \frac{(RSS_r - RSS_{ur}) / r}{RSS_{ur} / (n - k_{ur})}, RSS_{ur} = RSS_1 + RSS_2.
\]
\end{solution}

