
\begin{question}
Для выбора между линейной и полулогарифмической моделями (где EARNINGS --- почасовая заработная плата в \$, S --- длительность обучения, ASVABC - результаты тестов, характеризующие успеваемость) был проведен тест Дэвидсона, Уайта и МакКиннона и получены следующие результаты:

\begin{tabular}{l c c c c }
\toprule
& Зависимая: $Y$ & Зависимая: $\ln Y$  \\
\midrule
(Intercept) & $-26.148$     &  $-1.941$    \\
& $(4.17)$     &  $(3.2499)$   \\
S           &  $2.008$     &  $0.087$ \\
& $(0.276)$    &  $(0.035)$ \\
ASVABC      &  $0.393$    &   $0.017$ \\
& $(0.079)$   &   $(0.007)$ \\
lin\_add    &  $-15.373$   &     \\
&   $(5.984)$  &     \\

semilog\_add    &              & $-0.029$    \\
&              & $(0.065)$  \\
\midrule
$R^2$        & 0.2071         & 0.2212        \\
F            & 46.59          & 50.74          \\
Adj. $R^2$   & 0.2027         & 0.2168        \\
Num. obs.    & 540            & 540           \\
RSS          & 90975.57          & 148.1      \\
$\hat\sigma$ & 13.04        & 0.5256      \\
\bottomrule
\end{tabular}

Где \(\text{lin\_add} = \ln(\hat y) - \widehat{\ln} Y, \text{semilog\_add} = \hat y - \exp(\widehat{\ln} Y)\) и в скобках указаны стандартные ошибки.

На уровне значимости 5\% можно сделать вывод, что
\begin{answerlist}
  \item Лучше линейная модель
  \item Между линейной и полулогарифмической моделями нет статистической разницы
  \item Лучше линейная в логарифмах модель
  \item Невозможно выбрать лучшую модель
  \item Лучше полулогарифмическая модель
\end{answerlist}
\end{question}

\begin{solution}
\begin{answerlist}
  \item Bad answer :(
  \item Bad answer :(
  \item Bad answer :(
  \item Bad answer :(
  \item Good answer :)
\end{answerlist}
\end{solution}

