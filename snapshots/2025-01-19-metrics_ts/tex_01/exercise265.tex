
\begin{question}
Эмманюэль и Владимир оценили зависимость стоимости подержанных Пежо (одной серии) \(Y\)
от пробега \(X\) (измеряемого в км) с помощью модели парной регрессии \(Y = \alpha + \beta X + u\) по по одной и той же выборке,
однако Эмманюэль измерял стоимость машин в евро, а Владимир -- в рублях, 1 евро = 65 рублей.
Оценки МНК коэффициента наклона регрессии, полученные Эмманюэлем \(\beta_E\) и Владимиром \(\beta_B\) связаны следующим образом:
\begin{answerlist}
  \item \(\hat \beta_B = \hat \beta_E\)
  \item \(\hat \beta_B = 4225 \hat \beta_E\)
  \item \(\hat \beta_E = 65 \hat \beta_B\)
  \item \(\hat \beta_B = 65 \hat \beta_E\)
  \item \(\hat \beta_E = 4225 \hat \beta_B\)
\end{answerlist}
\end{question}

\begin{solution}
\begin{answerlist}
  \item Bad answer :(
  \item Bad answer :(
  \item Bad answer :(
  \item Good answer :)
  \item Bad answer :(
\end{answerlist}
\end{solution}

