
\begin{question}
Элеонора исследует зависимость цены номера в отеле от звёздности отеля, \(star\),
(от 1 до 3 звёзд) и расстояния до моря, \(dist\).
Элеонора хочет оценить модель вида
\(price_i = \beta_1 + \beta_2 star_i + \beta_3 dist_i + u_i\).
Чтобы считаться богиней эконометрики Элеоноре стоит
\begin{answerlist}
  \item заменить переменную \(star_i\) на дамми-переменные \(one_i\), \(two_i\) и \(three_i\), равные 1 для отелей с одной, двумя и тремя звёздами соответственно
  \item заменить переменную \(star_i\) на дамми-переменные \(one_i\) и \(two_i\), равные 1 для отелей с одной и двумя звёздами соответственно
  \item добавить дамми-переменные \(one_i\), \(two_i\) и \(three_i\), равные 1 для отелей с одной, двумя и тремя звёздами соответственно
  \item добавить в модель переменную \(z_i = star^2_i\), так как эффект звёздности наверняка нелинейный
  \item добавить в модель переменную \(z_i = star_i \cdot dist_i\)
  \item использовать МНК для оценки данной модели
\end{answerlist}
\end{question}

\begin{solution}
\begin{answerlist}
  \item Bad answer :(
  \item Good answer :)
  \item Bad answer :(
  \item Bad answer :(
  \item Bad answer :(
  \item Bad answer :(
\end{answerlist}
\end{solution}

