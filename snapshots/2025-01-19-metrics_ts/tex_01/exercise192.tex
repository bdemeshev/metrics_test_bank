
\begin{question}
Имеются данные по 100 работникам: затраты на проезд в общественном транспорте (\(E_i\), руб.),
количество часов работы в день (\(WH_i\), руб.),
количество часов отдыха в день (\(LH_i\), руб.) и
количество часов сна в день (\(SH_i\), руб.).
Считая, что всё время суток распределяется между трудом, сном и отдыхом,
оценка регрессии в виде
\[
E_i = {\beta_1} + {\beta_2}WH_i + {\beta _3}LH_i + {\beta _4}SH_i + u_i
\]
приведет к тому, что
\begin{answerlist}
  \item МНК--оценки получить не удастся
  \item МНК-оценки параметров регрессии будут несмещенными и эффективными
  \item коэффициент детерминации \(R^2\) окажется отрицательным
  \item МНК-оценки параметров окажутся неэффективными в классе линейных и несмещённых
  \item МНК-оценки параметров окажутся смещёнными
\end{answerlist}
\end{question}

\begin{solution}
\begin{answerlist}
  \item Good answer :)
  \item Bad answer :(
  \item Bad answer :(
  \item Bad answer :(
  \item Bad answer :(
\end{answerlist}
\end{solution}

