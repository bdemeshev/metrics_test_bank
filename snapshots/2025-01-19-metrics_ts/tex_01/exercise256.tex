
\begin{question}
Что может произойти с \(R^2\) при добавлении нового наблюдения в модель множественной регрессии с константой?
\begin{answerlist}
  \item Коэффициент \(R^2\) может остаться неизменным или упасть.
  \item Коэффициент \(R^2\) может остаться неизменным или вырасти.
  \item Коэффициент \(R^2\) обязательно упадёт.
  \item Коэффициент \(R^2\) обязательно вырастет.
  \item Коэффициент \(R^2\) может измениться в любую сторону.
\end{answerlist}
\end{question}

\begin{solution}
Достаточно поэкспериментировать с парной регрессией или картинкой на плоскости.
\end{solution}

