
\begin{question}
Петя и Вася проверяют гипотезу \(\beta_x=0\) против альтернативной \(\beta_x \neq 0\) по одним и тем же данным,
одни и тем же способом.
Единственная разница в том, что Петя использует уровень значимости \(\alpha = 0.01\), а Вася --- \(0.02\).

Рассмотрим 4 ситуации:

А. Петя отверг \(H_0\), Вася отверг \(H_0\).

Б. Петя отверг \(H_0\), Вася не отверг \(H_0\).

В. Петя не отверг \(H_0\), Вася отверг \(H_0\).

Г. Петя не отверг \(H_0\), Вася не отверг \(H_0\).

Какие из этих ситуаций возможны?
\begin{answerlist}
  \item только В, Г
  \item только А и Г
  \item все возможны
  \item только А, Б, Г
  \item только А, Б
  \item только А, В и Г
\end{answerlist}
\end{question}

\begin{solution}
Более высокая ошибка первого рода означает более частое отвержение \(H_0\).
\end{solution}

