
\begin{question}
Рассмотрим уравнение
\[
y_t = 3 + 0.7 y_{t-1} + u_t + 0.6 u_{t-1},
\]
где \((u_t)\) --- белый шум.

Определите, являются ли верными утверждения A и B.

A: уравнение имеет одно стационарное решение вида \(MA(\infty)\) относительно \((u_t)\).

B: для \(MA\)-части уравнения выполнено условие обратимости.
\begin{answerlist}
  \item A верно, B верно.
  \item A верно, B неверно.
  \item A неверно, B неверно.
  \item A неверно, B верно.
\end{answerlist}
\end{question}

\begin{solution}
Вспомните определение обратимости и теорему о виде стационарных решений.
\end{solution}

