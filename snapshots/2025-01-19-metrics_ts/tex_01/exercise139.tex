
\begin{question}
Рассмотрим два процесса.

Первый, \(y_t = 5 + u_t + 2 u_{t-1}\), где \((u_t)\) --- белый шум с дисперсией \(8\).

Второй, \(x_t = 5 + v_t + k v_{t-1}\), где \(k\neq 2\) и \((v_t)\) --- белый шум с дисперсией \(\sigma^2\).

Известно, что у процессов \((y_t)\) и \((x_t)\) одинаковые автоковариационные функции.
Найдите \(\sigma^2\) с точностью до двух знаков после десятичной точки.
\end{question}

\begin{solution}
Вспомните определения автоковариационной функции.
\end{solution}

