
\begin{question}
Рассмотрим алгоритм случайного леса для задачи регрессии.

Какую целевую переменную «учится» прогнозировать дерево номер 4?
\begin{answerlist}
  \item разницу между исходной переменной \(y_t\) и суммой прогнозов первых трёх деревьев
  \item нет верного ответа
  \item разницу между исходной переменной \(y_t\) и суммой прогнозов первых трёх деревьев, домноженной на темп обучения
  \item разницу между исходной переменной \(y_t\) и прогнозом третьего дерева
  \item исходную переменную \(y_t\)
  \item разницу между исходной переменной \(y_t\) и прогнозом третьего дерева, домноженным на темп обучения
\end{answerlist}
\end{question}

\begin{solution}
Посмотрите описание случайного леса.
\end{solution}

