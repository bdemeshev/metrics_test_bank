
\begin{question}
Какая зависимость математического ожидания исходного процесса от времени предполагается в нулевой гипотезе \(KPSS\)-теста с константой?
\begin{answerlist}
  \item \(\mathbb{E}(y_t) = \mu + \alpha t\)
  \item \(\mathbb{E}(y_t)\) строго монотонна
  \item нет верного ответа
  \item \(\mathbb{E}(y_t) = \mu + \alpha t + \beta t^2\)
  \item \(\mathbb{E}(y_t)\) строго возрастает
  \item \(\mathbb{E}(y_t) = \mu\)
\end{answerlist}
\end{question}

\begin{solution}
Посмотрите предпосылки тестов.
\end{solution}

