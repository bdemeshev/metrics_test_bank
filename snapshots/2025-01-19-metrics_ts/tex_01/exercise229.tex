
\begin{question}
Рассмотрим модель \(Y = X \beta + u\). Условия теоремы Гаусса-Маркова выполнены, причём \(\mathrm{Var}(u_i) = \sigma^2_{u}\), \(\hat y = PY\), \(P = X (X'X)^{-1} X'\) и \(I\) - единичная матрица. Ковариационная матрица случайного вектора \(e=Y-\hat y\) равна
\begin{answerlist}
  \item \(\sigma^2_{u} (I+P)\)
  \item \(\sigma^2_{u} (P-I)\)
  \item \(\sigma^2_{u} P\)
  \item \(\sigma^2_{u} (I - P)\)
  \item \(\sigma^2_{u} I\)
\end{answerlist}
\end{question}

\begin{solution}
\begin{answerlist}
  \item Bad answer :(
  \item Bad answer :(
  \item Bad answer :(
  \item Good answer :)
  \item Bad answer :(
\end{answerlist}
\end{solution}

