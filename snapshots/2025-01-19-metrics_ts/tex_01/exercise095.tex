
\begin{question}
У Маши две монетки: медная и серебряная. Маша подкинула каждую монетку 100 раз. Затем с помощью метода максимального правдоподобия
Маша трижды оценила вероятность выпадения орла: для медной монетки, для серебряной и по объединённой выборке.
Значения функции правдоподобия равны \(\ell_{copper} = -300\), \(\ell_{silver}=-200\) и \(\ell_{common} = -510\).

\(LR\) статистика, проверяющая гипотезу о равенстве вероятностей выпадения орла для двух монеток, равна
\begin{answerlist}
  \item \(10\)
  \item \(1010\)
  \item \(5\)
  \item \(500\)
  \item \(20\)
\end{answerlist}
\end{question}


